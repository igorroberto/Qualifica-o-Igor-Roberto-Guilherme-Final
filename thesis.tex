% ------------------------------------------------------------------------
% ------------------------------------------------------------------------
% ICMC: Modelo de Trabalho Acadêmico (tese de doutorado, dissertação de
% mestrado e trabalhos monográficos em geral) em conformidade com 
% ABNT NBR 14724:2011: Informação e documentação - Trabalhos acadêmicos -
% Apresentação
% ------------------------------------------------------------------------
% ------------------------------------------------------------------------

% Opções: 
%   Qualificação         = qualificacao 
%   Curso                = doutorado/mestrado
%   Situação do trabalho = pre-defesa/pos-defesa (exceto para qualificação)
\documentclass[qualificacao,mestrado]{packages/icmc}

% ---------------------------------------------------------------------------
% Pacotes Opcionais
% ---------------------------------------------------------------------------
\usepackage{rotating}           % Usado para rotacionar o texto
\usepackage[all,knot,arc,import,poly]{xy}   % Pacote para desenhos gráficos
% Este pacote pode conflitar com outros pacotes gráficos como o ``pictex''
% Então é necessário usar apenas um dos pacotes conflitantes
\newcommand{\VerbL}{0.52\textwidth}
\newcommand{\LatL}{0.42\textwidth}
% Comando simples para exibir comandos Latex no texto
\newcommand{\comando}[1]{\textbf{$\backslash$#1}}
% ---------------------------------------------------------------------------


% ---
% Informações de dados para CAPA e FOLHA DE ROSTO
% ---
% Tanto na capa quanto nas folhas de rosto apenas a primeira letra da primeira palavra (ou nomes próprios) devem estar em letra maiúscula, todas as demais devem ser em letra minúscula.

\tituloPT{Uma proposta de sistema baseado em ontologias para o gerenciamento de filas de espera na atenção básica de saúde}

\tituloEN{A proposal for an ontology based system for queue management in basic health care}
\autor[Guilherme, I. R.]{Igor Roberto Guilherme}
\genero{M} % Gênero do autor (M = Masculino / F = Feminino)
\orientador[Orientador]{Prof. Dr.}{Dilvan de Abreu Moreira}
%\coorientador{Prof. Dr.}{Fulano de Tal}
\curso{CCMC}
\data{03}{03}{2019} % Data do depósito
\idioma{PT} % Idioma principal do documento (PT = português / EN = inglês)
% ---


% ---
% RESUMOS
% ---

% Resumo em PORTUGUÊS
% conter no máximo 500 palavras
% conter no mínimo 1 e no máximo 5 palavras-chave
\textoresumo[brazil]{
O longo tempo de espera para o atendimento médico é um dos principais problemas do Sistema Único de Saúde (SUS). O tempo que um paciente tem que aguardar pelo atendimento, além de causar sofrimento, pode agravar a situação da sua enfermidade. A regulação dessas filas é responsabilidade de cada município, que pode adotar inúmeras formas para agendar um atendimento, não tendo a obrigatoriedade de expor seus critérios de regulação. Em muitos casos, são funcionários administrativos que realizam a gestão e ordenamento dessas filas, sem ter um conhecimento aprofundado de cada caso para gerir a fila de forma correta. O computador pode ajudar no gerenciamento dessas filas, utilizando critérios definidos por especialistas. Para tal, é necessário que os dados sobre os pacientes estejam semanticamente representados, para a interpretação e gestão por parte das máquinas. A Web Semântica, através das ontologias, permite a representação semântica do conhecimento, possibilitando que as máquinas interpretem e auxiliem na gestão dessas filas.
O presente trabalho propõe o uso das tecnologias da Web Semântica, mais precisamente ontologias OWL, para a representação dos indicadores para classificação de prioridades para o atendimento em uma especialidade médica. Como estudo de caso, uma ontologia OWL para o Hospital das Clínicas da Faculdade de Medicina de Marília (FAMEMA) e um protótipo que utilize essa ontologia e gerencie uma fila de retorno para uma especialidade médica serão implementados. Almeja-se que o modelo proposto se mostre como uma boa alternativa ao modelo atual de gestão de filas, e que, após uma validação por especialistas (médicos), o gerenciamento computacional seja mais eficiente se comparado ao gerenciamento realizado atualmente no HC.
}{Fila de espera na saúde, Web Semântica, Sistema Único de Saúde, Ontologia}

\textoresumo[english]{The long wait time for medical care is one of the main problems of the Unified Health System (SUS). The time that a patient has to wait to receive health care, besides causing suffering, can aggravate the situation of his illness. The regulation of these queues is the responsibility of each municipality, which can adopt a number of ways to schedule the service, not having to state its regulatory criteria. In many cases, administrative employees manage and order these queues without having an in-depth knowledge of each patient case to manage it properly.  Computers can help manage these queues using criteria defined by experts. For that to happen, it is necessary that patient data be semantically represented, for its interpretation and management by machines. Semantic Web technologies allow the representation of knowledge using ontologies, allowing machines to interpret and assist in the management of these queues. The present work proposes the use of Semantic Web technologies, more precisely OWL ontologies, for the representation of indicators for priority classification of patients attending a medical specialty. As a case study, an OWL ontology for the Hospital das Clínicas of Marília Medical School (FAMEMA) and a prototype that uses this ontology to manage a return queue for a medical specialty will be implemented. It is hoped that the proposed model will prove to be a good alternative to the current queuing management model, and that, after validation by specialists (physicians), the computational management will be considered more eficient when compared to the current management performed in the HC.
}{Waiting queue in health, Semantic Web,
Health Unic System, Ontology}
    
% ---
% Configurações de aparência do PDF final
% ---
\hypersetup{
	colorlinks=true     % false: boxed links; true: colored links
}
% --- 

% ----------------------------------------------------------
% ELEMENTOS PRÉ-TEXTUAIS
% ----------------------------------------------------------

% Inserir a ficha catalográfica
% \incluifichacatalografica{tex/ficha-catalografica.pdf}

% DEDICATÓRIA / AGRADECIMENTO / EPÍGRAFE
%\textodedicatoria*{tex/pre-textual/dedicatoria}
%\textoagradecimentos*{tex/pre-textual/agradecimentos}
%\textoepigrafe*{tex/pre-textual/epigrafe}

% Inclui a lista de figuras
\incluilistadefiguras

% Inclui a lista de tabelas
\incluilistadetabelas

% Inclui a lista de quadros
% \incluilistadequadros

% Inclui a lista de algoritmos
%\incluilistadealgoritmos

% Inclui a lista de códigos
%\incluilistadecodigos

% Inclui a lista de siglas e abreviaturas
\incluilistadesiglas

% Inclui a lista de símbolos
%\incluilistadesimbolos

% ----
% Início do documento
% ----
\begin{document}
% ----------------------------------------------------------
% ELEMENTOS TEXTUAIS
% ----------------------------------------------------------
\textual

\chapter{Introdução}
\label{chapter:introducao}
A fila de espera faz parte do nosso cotidiano, enfrentamos filas para diversos objetivos, seja no supermercado, em Bancos e até mesmo para um atendimento médico. No caso dos serviços de saúde, a longa espera pode propiciar o sofrimento do paciente, reduzir as possibilidades de cura, permitir o agravamento das enfermidades ou a extensão das sequelas e até determinar risco de morte \cite{Gazzinelli2014}. 

No Sistema Único de Saúde(SUS), marcar consulta com especialista é o maior problema para os brasileiros \cite{CFM2018}. Não há um modelo claro de regulação do SUS, ficando a cargo de cada município gerir as filas de espera para os atendimentos especializados. A gestão destas filas ocorre de maneira manual, em muitos casos por funcionários das unidades de saúde que não obtém de um conhecimento  amplo para identificar e comparar casos de cada enfermidade. Neste cenário, os funcionários adotam um agendamento das consultas, predominantemente, pela ordem de chegada das fichas de encaminhamento na recepção, exceto quando o médico sinaliza a urgência no atendimento do caso, como relatado no trabalho de \cite{SOUZA2014}.

Apesar de fazer parte do dia-a-dia dos médicos que trabalham em serviço público, o problema da fila de espera é muito pouco abordado pela comunidade médica e científica, talvez por parecer tratar-se de uma discussão que não pertence aos meios acadêmicos e sim às instâncias governamentais. Todavia é preciso salientar que o acesso equitativo, justo e universal aos serviços de saúde deve ser uma preocupação constante não só do governo como de todos os profissionais envolvidos no atendimento à rede pública. Muito pode ser feito em nível local a respeito das filas de espera. \cite{KRISHNAMURTI2005}

O tempo de espera é um indicador da qualidade dos serviços, por estar relacionado à capacidade de resposta do sistema às necessidades de atenção à saúde da população \cite{Gazzinelli2014}. 

A tecnologia é muito importante para a gestão de saúde,
ultrapassando o processamento-padrão de dados para funções administrativas comuns em todas as organizações, tais como recursos humanos, folhas de pagamento, sistemas de contabilidade, entre outros, e agora desempenha um papel fundamental tanto no cuidado ao paciente, na interpretação do eletrocardiograma, como em escalas de trabalho, prescrição, relatório de resultados e sistemas de prevenção \cite{PINOCHET2014}. O computador pode auxiliar no processamento de informações de saúde, para \cite{MATTOS1978}, as tarefas realizadas pelo computador, desde que corretamente programado, são praticamente isentas de erro, o que não ocorre no caso do processamento manual (realizado por pessoas). O trabalho ainda relata que o principal motivo dessa perfeição é que a máquina não é sensível aos fatores que reduzem a qualidade do trabalho humano, quais sejam:

  \begin{enumerate}
    	\item a fadiga - que faz com que o número de erros cometidos involuntariamente aumente ao fim do turno de trabalho;
        \item Dos problemas pessoais - uma preocupação com a família, por exemplo, perturba bastante o bom desempenho do funcionário;
        \item o desajuste com a empresa - um empregado mal pago, ou ameaçado de ser despedido, "vinga-se" da empresa sabotando o sistema de informações do qual participa;
        \item interesses pessoais - o funcionário procura adulterar o sistema em beneficio próprio. Tal é o caso, por exemplo, da emissão de uma ordem de serviço para a instalação de um telefone em uma loja comercial, passando na frente de centenas de outros clientes, por um empregado" que recebe uma propina do proprietário dessa loja;
        \item desorganização e falta de planejamento - a incapacidade de um chefe pode desorientar seus subordinados, reduzindo a qualidade de seu serviço.
    \end{enumerate}
    
O trabalho apresenta os itens com uma visão mais empresarial, porém, é possível ter um entendimento com um olhar sobre os funcionários gestores das filas de espera para o encaminhamento especializado. O item 4 por exemplo, é perfeitamente comparável a de funcionários que priorizam o atendimento de amigos e parentes na gestão destas filas, conforme citado no trabalho de \cite{KRISHNAMURTI2005}

A Web Semântica é uma destas tecnologias que podem auxiliar processos na área da saúde. Atrvés da OWL - Ontology Web Language, criada pela World Wide Web Consortium (W3C), é possível tornar as informações, antes apenas compreensível por humanos, em informações passíveis de interpretação por computadores. O gerenciamento das filas de espera por parte dos computadores, pode contribuir para se alcançar filas isentas de fatores humanos (descritos na lista anterior), ainda possibilitar a redução do tempo de espera para os pacientes que precisam de um atendimento imediato de acordo com suas características físicas e sociais. 

Como uma instância destas lacunas é possível citar o caso do Hospital das Clínicas da Faculdade de Medicina de Marília(FAMEMA), onde o problema da fila de espera ocorre em casos de retorno de um paciente a uma especialidade, cuja fila é denominada como "Demanda". Na Demanda, é utilizado o método tradicional de fila(First In, First Out - FIFO), gerando "filas sem fim", nas quais os pacientes chegam a esperar mais de oito anos por um atendimento. No gerenciamento da Demanda, ainda não é levado em consideração o quadro clínico do paciente, tais como o diagnóstico, gravidade, idade e a capacidade de um paciente se recuperar de uma enfermidade no caso da urgência no seu retorno. A demanda é distribuída por profissional e não por especialidade, necessitando que em casos de desligamento de algum profissional, a distribuição manual da fila para outros médicos.

Um efeito colateral do gap entre a entrada do paciente na Demanda e o seu efetivo atendimento é o absenteísmo. Por vários motivos: o problema inicial para o qual o paciente procurou atendimento não existe mais, foi atendido por outro estabelecimento/profissional de saúde, o paciente não foi localizado, o paciente veio a óbito, entre outros. 



\section{Objetivos}
    O objetivo deste trabalho é analisar as filas de espera do Sistema Único de Saúde (SUS), mais precisamente no encaminhamento do atendimento básico para o atendimento especializado, e propor um modelo computacional para o gerenciamento dessas filas, com o intuito de torná-las mais coerentes com a prioridade de atendimento definida por especialistas (médicos), sem a necessidade da alocação de um profissional dedicado à essa função. 
    
    Ainda pretende-se obter filas mais justas,  minimizando a possibilidade de indivíduos serem atendidos prioritariamente por terem acesso aos gestores dessas filas, como citado no trabalho de \citeonline{KRISHNAMURTI2005}. Objetiva-se que os critérios de ordenamento da fila sejam mais transparentes, e que o posicionamento de cada paciente seja feito com isenção, levando em consideração apenas a situação física e social de cada um.

\subsection{Objetivos específicos}

	Para alcançar os objetivos apresentados na seção anterior, foram definidos os objetivos específicos:
    
    \begin{itemize}
    	\item Criação de uma ontologia de domínio no formato OWL que represente indicadores para classificação de prioridades para o atendimento de uma especialidade médica.
        \item Desenvolvimento de um protótipo que utilize a ontologia e, de acordo com critérios médicos, gerencie uma fila de espera.
        \item Acompanhamento da avaliação dos médicos do Hospital das Clínicas da Faculdade de Medicina de Marília do protótipo com o intuito de verificar como o gerenciamento computacional se  sobressai se comparado ao gerenciamento atual do HC.
    \end{itemize}


\section{Organização do Trabalho}

	A presente dissertação está organizada da seguinte forma:
	
    \begin{itemize}
    	\item Capítulo 2: Apresenta os conceitos relevantes que estão relacionados com o este trabalho;
        \item Capítulo 3: Expõe as características dos trabalhos relacionados;
        \item Capítulo 4: Descreve a proposta da pesquisa, a metodologia utilizada e o cronograma detalhado de cada atividade.
    \end{itemize}

\chapter{Referencial Teórico}
\label{chapter:referencial-teorico}
Neste capítulo são apresentados em cada subseção conceitos importantes para a construção e melhor compreensão deste trabalho.

\section{Fila de espera no Sistema Único de Saúde}
    O Sistema Único de Saúde (SUS) é um dos maiores sistemas públicos de saúde do mundo, sendo o único a garantir assistência integral e completamente gratuita para a totalidade da população \cite{SOUZA2002}. Diferentemente de outros países, no Brasil mesmo quem opta por um plano de saúde privado tem o direito de ser atendido em qualquer unidade do Sistema Único de Saúde.
   
   \subsection{Financiamento do SUS}
   
     O financiamento do SUS é uma responsabilidade comum entre os governos nacional, estadual e municipal que realizam a destinação de porcentagens diferentes de seu orçamento para o SUS, seguindo as seguintes regras \cite{CONASS}:
    \begin{itemize}
        \item União: A Emenda Constitucional n. 86 de 17 de março de 2015 definiu que a partir de 2016 a União aplicará, anualmente, em ações e serviços públicos de saúde, o montante não inferior a 15\% correspondente ao valor da Receita Corrente Líquida (RCL) do respectivo exercício financeiro.
        \item  Estados e Distrito Federal: No mínimo 12\% da arrecadação dos impostos é destinada a ações e serviços públicos de saúde, sendo deduzidas as parcelas que forem transferidas aos municípios;
        \item Municípios e o Distrito Federal: Anualmente aplicam em ações e serviços públicos de saúde, no mínimo, 15\% da arrecadação dos impostos.
    \end{itemize} 

   
   	\subsection{Funcionamento do SUS}
   	
    O Governo Federal tem o dever de fiscalizar e elaborar mecanismos de apoio para que estados e municípios possam oferecer serviços de saúde.

   É na instância municipal que o paciente dá entrada no sistema, por meio da \sigla{UBS}{\textit{Unidade Básica de Saúde}} ou pela equipe da \sigla{USF}{\textit{Unidade Saúde da Família}} que são profissionais que acompanham um número de famílias em uma determinada área geográfica. Na UBS é realizado o atendimento com hora marcada e deve sempre haver médicos de três especialidades: clínico geral, pediatra e ginecologista. Nessas unidades é oferecido um primeiro atendimento, afim de realizar uma triagem e quando necessário solicitar o encaminhamento para as demais especialidades \cite{CONTE2017}.
   
   Na Figura 1 abaixo, é apresentado o fluxo pelo qual o paciente é submetido dentro do SUS desde o primeiro contato até um atendimento especializado ou internação:
    
     \begin{figure}[htbp]
        	\centering
            \caption{Fluxograma de funcionamento do SUS}
            \label{fig:images/fluxograma-trajetoria-usf-pe}
            \includegraphics[width=0.9\linewidth]{images/funcionamento-sus.png}
            \fdireta{SECRETARIAMG}
        \end{figure}
    
    Ainda na UBS, caso o paciente necessite de um atendimento emergencial, a \sigla{UPA}{\textit{Unidade de Pronto Atendimento}} é quem o recebe. As UPAs são unidades de complexidade intermediária entre  as UBSs e a emergência dos hospitais, portanto, servem para "desafogar" as filas dos hospitais. Segundo o Ministério da Saúde, 97\% dos casos que chegam as UPAs são solucionados \cite{BRASIL2012}. Os pacientes que procuram as UPAs são avaliados de acordo com a classificação de risco, ou seja, os casos mais graves terão prioridade.
    
    Pode se entender como atendimento secundário, todo atendimento especializado oferecido por hospitais e clínicas. Nesse nível de atendimento,  espera-se que os profissionais estejam preparados para realizar procedimentos de média complexidade, conduzindo o tratamento de quadros que comprometem o bem-estar e a qualidade de vida dos pacientes de forma aguda ou crônica.
    
    O atendimento terciário é o ultimo nível de atenção em saúde no Brasil, por tanto, o mais complexo. É nesse nível que são realizadas cirurgias e exames mais invasivos, que exigem a mais avançada tecnologia em saúde. Ou seja, o nível terciário visa à garantia do suporte mínimo necessário para preservar a vida dos pacientes nos casos em que a atenção no nível secundário não foi suficiente para isso.

     \subsection{Filas de espera}
    
    O caminho que um paciente percorre até o efetivo atendimento pode ser longo e muitas vezes acaba precisando retornar ao início do fluxo de atendimento. O trabalho de \citeonline{SOUZA2014}, avaliou as condições de acesso integral dos usuários a partir do caminho percorrido desde a atenção básica até a especializada na rede assistencial de Recife - PE. O fluxograma do caminho percorrido pelos pacientes é exibido na Figura 2:
    
     \begin{figure}[htbp]
        	\centering
            \caption{Fluxograma descritor do acesso do usuário da atenção primária à atenção especializada.}
            \label{fig:images/fluxograma-trajetoria-usf-pe}
            \includegraphics[width=0.8\linewidth]{images/fluxograma-trajetoria-usf-pe}
            \fdireta{SOUZA2014}
        \end{figure}
        
    Quando um médico de família encaminha um usuário ao especialista, esse procura a recepção da USF para entregar a sua ficha de encaminhamento ao profissional responsável pela marcação das consultas que, por sua vez, faz o contato telefônico com a Central de Regulação do Recife para o agendamento.
    
    A demora no agendamento das consultas especializadas é uma das barreiras de acesso para o atendimento integral da população. No estudo, ainda é citado que a inexistência de critérios definidos para a escolha do serviço de referência, no qual os usuários serão agendados, é outro aspecto desordenador do acesso. Essa definição cabe ao pessoal administrativo, ou os ‘marcadores das consultas’. Outro aspecto que chama a atenção, como elemento desordenador, é o fato de o agendamento das consultas ser realizado, predominantemente, pela ordem de chegada das fichas de encaminhamento na recepção. Com exceção dos casos em que o próprio médico sinaliza a ‘urgência’ ou ‘prioridade’ dos pacientes, é o responsável pela marcação das consultas que tenta priorizar
    quem deverá ter acesso e agendamento prioritário. Esse aspecto demonstra que a análise de risco clínico não se coloca como atividade consolidada no processo de trabalho das equipes de saúde da família \cite{SOUZA2014}.
    
    Um outro problema abordado na literatura quando se trata de atendimento especializado, é a questão do absenteísmo. No trabalho de \citeonline{URSULA2018} é realizada uma análise do impacto da fila de espera no absenteísmo de exames e consultas. O estudo revela que as maiores
    probabilidades de ocorrência de absenteísmo são obtidas, de modo geral, em
    procedimentos que ultrapassam os 60 dias de espera. O trabalho afirma que a redução do tempo de espera para menos de 60 dias possibilitou a diminuição significativa do absenteísmo, o que repercute na diminuição do agravamento de doenças e nos custos ao sistema de saúde. 
    
    A fila de espera não é um entrave apenas no SUS, é um problema em cerca da metade dos países da \sigla{OECD}{\textit{Organisation for Economic Co-operation and Development}}  \cite{SICILIANI2004}. Recursos financeiros escassos, quantidade de vagas, gestão e estrutura são também presentes em outros países, como o \sigla{SNS}{\textit{Sistema Nacional de Saúde espanhol}}. No trabalho de \citeonline{CONILL2011}, para o enfrentamento dessas barreiras no acesso ao sistema de saúde, foram realizadas iniciativas que proporcionassem a continuidade assistencial. Foi adotada a medida de circuitos preferenciais. Essa estratégia serve para encaminhar da atenção primária com preferência os usuários com suspeitas de alguma etiologia específica.
    
    Uma parceria entre o Fórum Espanhol de Pacientes em conjunto com a Universidade de Harvard (EUA) desenvolveu uma pesquisa com 3.010 cidadãos para avaliar a confiança dos espanhóis no sistema nacional de saúde. Entre outros aspectos, esse trabalho revelou que, sem distinção de classe social, área geográfica ou densidade demográfica, as listas de espera são o principal problema dos serviços de saúde espanhóis para 78\% dos entrevistados \cite{LOPEZ2007}.
    
    Vale salientar que as filas de espera do atendimento básico para o especializado, não são as únicas onde o paciente encontra problemas, filas como de urgências e cirurgias também são dificuldades reais no SUS. O trabalho de \citeonline{KRISHNAMURTI2005} trás uma abordagem das filas de espera para cirurgias otorrinolaringológicas no SUS. O trabalho descreve que o maior afunilamento acontece na obtenção da consulta no ambulatório de otorrinolaringologia, ou seja, no atendimento especializado (seta nº 3, na figura 3).
    
    \begin{figure}[htbp]
        	\centering
            \caption{Fluxograma de atendimento - as setas tracejadas representam os pontos críticos no processo de obtenção do tratamento.}
            \label{fig:images/fluxo-fila-cirurgia-otorrino}
            \includegraphics[width=0.9\linewidth]{images/fluxo-fila-cirurgia-otorrino.png}
            \fdireta{KRISHNAMURTI2005}
        \end{figure}
        
    O trabalho ainda discute um tema importante, ao questionar se os indivíduos dessas filas de espera realmente passaram por vias regulares e justas para o atendimento. Muitos que têm a possibilidade, procuram outras formas de obter a consulta, surgem então os pedidos de "consulta extras". São os chamados "PAFs" ou "PAFUNCIOs" tão conhecidos do médico que trabalha em serviço público: "Parente ou Amigo de Funcionário".

    No artigo, o autor relata que ocorre uma pressão não só diretamente sobre os médicos, mas também sobre enfermeiros, auxiliares de enfermagem e demais funcionários. Os funcionários responsáveis pela marcação das consultas também são pressionados, de modo que é preciso atentar para o fato de que mesmo o paciente que parece ter tido a sua consulta marcada por vias regulares pode ter sido beneficiado nessa marcação. Na opnião do autor, já há casos (crimes, citados por ele) de pessoas e funcionários que vendem essas consultas ou mesmo um lugar na fila da triagem do hospital. O médico, sem saber, passa a ser o instrumento de um lucrativo negócio: o do agenciamento da medicina pública \cite{KRISHNAMURTI2005}.
    
    Esses pacientes, independente da gravidade de suas queixas e do seu direito inquestionável ao atendimento, estão na realidade "furando" uma fila virtual de espera. Pode-se chegar a extremos em que o número de pessoas "furando" a fila é tal a ponto de criar uma fila paralela, com fluxo contínuo, enquanto a fila principal e legítima permanece quase parada. Concluindo, o trabalho ainda deixa claro que é necessário o enfrentamento da questão, e que os pacientes devem ser informados sobre sua condição, de preferência por escrito, da indicação cirúrgica, da existência da fila, dos critérios de prioridade nessa fila, da quantidade de pessoas à sua frente e de uma estimativa do tempo até sua cirurgia, deixando-se claro tratar-se apenas de uma estimativa.
    
    % aqui pode-se falar que a falta de critérios de entrada e priorização de atendimento permitem que pacientes possam "furar" a fila, novamente é complicado afirmar que isso ocorre...
    
    %Dilvan- O que o artigo diz? Se ele fala em crime ou furar fila,isso deve ser mantido. Igor,voce nao deve falar nada a mais do que esta no artigo.
    
    %Igor:  conforme relatado no parágrafo, são informações que o autor relata e foram apenas citadas neste trabalho, todas os termos "crimes", "furando fila" foram retirados da forma que é descrita no trabalho
  



    
  

\section{Web Semântica e Ontologia}

    Como apresentado até aqui, é necessário que sejam propostas novas soluções como forma de organizar os critérios de encaminhamento e priorização dos serviços básicos aos especializados. Mais que isso, possibilitar que o ordenamento e a gestão das listas de espera não fiquem refém de todos os fatores humanos associados a esse tipo de ação. Dessa forma, as tecnologias da Web Semântica podem auxiliar na gestão dessas listas, principalmente com a organização do conhecimento com as ontologias.
    
    Estima-se que na WEB existam cerca de 30 bilhões de documentos \cite{calderon2017deep}, que estão em diferentes formatos, como HTML, XML, PDF, CSV entre outros. Esses documentos são estruturados de diferentes formas para humanos, que tem a capacidade de interpretar os dados independente do formato do documento. Porém, essa diversidade de documentos acaba sendo um  grande obstáculo para o acesso, processamento e uso computacional desses dados.
    
    Em 2001, \citeonline{Berners2001} propuseram a Web Semântica, como forma de estruturar o conteúdo da web e permitir que o usuário possa realizar tarefas guiado por mecanismos chamados de agentes. Isso seria possível através da estruturação dos dados e da semântica presente em cada documento onde esses agentes virtuais conseguissem 'compreender'. Outro ganho que a proposta traria é de que as informações estariam interligadas e com fácil acesso, sendo dados interoperáveis.
    No mesmo texto de apresentação da Web Semântica publicado na revista Scientific American, os autores relatam que a Web Semântica não é uma Web separada, mas uma extensão da atual, na qual a informação recebe um significado bem definido, permitindo que computadores e pessoas trabalharem em cooperação.
    
   Para tornar a proposta aplicável, foi definida um arquitetura baseada em camadas, que pode ser melhor compreendida através da Figura 4.
   
     \begin{figure}[htb]
    	\centering
        \caption{Arquitetura da Web Semântica.}
    	\label{fig:semanticcake}
        \includegraphics[width=0.7\linewidth]{images/semantic-cake}
        \fdireta{Greenberg2003}
    \end{figure}
    
    Cada camada do chamado "bolo de noiva" da web semântica foi definida com o intuito de atingir o objetivo do modelo proposto.
    
    \begin{itemize}
    	\item \textit{Unicode}: é usado para representar qualquer caractere de maneira única, seja qual for o caractere e o idioma  que ele tenha sido escrito;
        \item \sigla{URI}{\textit{Uniform Resource Identifier}}: é uma representação única de um recurso;
        \item \sigla{XML}{\textit{eXtensible Markup Language}}: é uma linguagem de marcação e representação sintática de recursos de  maneira independente de plataforma a fim de agregar semântica aos documentos;
        \item \sigla{RDF}{\textit{Resource Description Framework}}: é um modelo padrão para intercâmbio de dados. Possui recursos que facilitam a mesclagem de dados, mesmo se os esquemas subjacentes forem diferentes, e suporta especificamente a evolução dos esquemas ao longo do tempo, sem exigir que todos os consumidores de dados sejam alterados;
        \item \textit{Ontology}: coleção de termos usados para descrever um domínio através das relações e classificações desses termos;
        \item \textit{Logic}: permite a definição de regras lógicas para deduzir e inferir novas informações. Essas regras são capazes de alterar dinamicamente a estrutura da ontologia;
        \item \textit{Proof}: provê mecanismos para averiguar a confiabilidade das fontes de informações;
        \item \textit{Trust}: representa o conhecimento validado e confiável;
        \textit{Digital Signature}: permite a integração de métodos de segurança que garantam a segurança da informação.
    \end{itemize}
    
    As ontologias exercem um papel fundamental na Web Semântica, pois são elas que permitem a relação de termos e conceitos de um domínio.
    
    \subsection{Ontologias}
    
        O termo ontologia pode ter mais de um significado, dependendo da área de conhecimento onde for utilizado. Seu surgimento se deu na filosofia, sendo uma teoria sobre a natureza da existência e envolve uma categorização muito ampla da realidade \cite{CHATEUABRIAND1998}.
        
        Ontologia na ciência da computação começou a ser utilizada no início dos anos 90, para organização de grandes bases de conhecimento, seu uso se dava para a construção de grandes bases interoperáveis e melhor estruturadas \cite{MOREIRA2004}. 
        
        Na Web Semântica, as ontologias são usadas para descrever os conceitos de um domínio e a relação entre eles. O W3C relata que as ontologias devem prover descrições para os seguintes tipos de conceitos \cite{breitman2005web}:
        
        \begin{itemize}
    	\item  Classes (ou coisas) nos vários domínios de interesse;
        \item  Relacionamento entre as classes;
        \item  Propriedades (ou atributos) que as classes podem ter; 
        \end{itemize}
        
   
	\subsection{\textit{Resource Description Framework} e \textit{Web Ontology Language}}
    
    	O RDF pode ser definido como um modelo padrão para intercâmbio de dados na Web. Ele é muito utilizado por conta de suas características que facilitam a fusão de dados, além de suportar a evolução dos esquemas sem a necessidade de que todos os dados sejam alterados, referenciando a relação de diferentes recursos \cite{W3C2014}.
    	Esse framework foi desenvolvido para descrever recursos na Web. Segundo \citeonline{Shadbolt2006}, o objetivo do RDF é prover uma representação minimalista do conhecimento da Web.
    	A estrutura do RDF é formada por triplas (Sujeito, Predicado, Objeto), como pode ser visto na Figura 5, sendo:
    	
        \begin{itemize}
        	\item Sujeito: recursos  identificados por URIs;
            \item Predicado: atributos ou relações que descrevem o sujeito e o relaciona a um objeto;
            \item Objeto: um outro recurso ou valor que se relaciona com o sujeito através do predicado;
        \end{itemize}
        
         \begin{figure}[htbp]
        	\centering
            \caption{Exemplo de grafo}
            \label{fig:owl2-profiles}
            \includegraphics[width=0.7\linewidth]{images/exemplo-tripla.png}
            \fdireta{SANTOSNETO2013}
        \end{figure}
    	
    	Através das triplas é possível representar o conhecimento de qualquer domínio. No caso da área da saúde, um exemplo de representação seria: uma pessoa, suas propriedades(idade, nome, sexo) e as possíveis relações que essa pessoa pode ter (doenças que teve, medicações aplicadas, exames realizados). 
        
        Como dito anteriormente, o RDF é um framework para descrever recursos, mas em qual formato? Qual estrutura? RDF pode ser serializado para diferentes formatos de acordo com a necessidade do uso. Na Figura 6 são apresentados os formatos de serialização do RDF, bem como a estrutura do código para cada formato e o propósito de uso de cada um.
         
         \begin{figure}[htbp]
        	\centering
            \caption{Formatos de serialização RDF}
            \label{fig:owl2-profiles}
            \includegraphics[width=1\linewidth]{images/serializacoes-rdf.png}
            \fdireta{isotani2015dados}
        \end{figure}
        
        Apesar do RDF permitir a descrição de ontologias, há algumas limitações quanto ao nível de expressividade que é possível alcançar, por consequência poucas inferências (capacidade que o computador deduza informações através de declarações estabelecidas) são possíveis. Por isso, foi desenvolvida uma linguagem mais expressiva denominada Web Ontology Language (OWL).
        
        A OWL também foi desenvolvida e é recomendada pelo W3C. Essa linguagem tem uma expressividade maior que o RDF ampliando suas possibilidades de representação de conhecimento. A primeira versão da OWL (1.0), possui 3 perfis de expressividade que permitem que aplicações, com diferentes objetivos, sejam construídas. Eles formam uma família de três variantes linguísticas de crescente poder expressivo: OWL Lite, OWL DL e OWL Full \cite{isotani2015dados}.
        
        Uma das limitações importantes da OWL 1 era a falta de um conjunto adequado de tipos de dados internos. Isso porque OWL depende do esquema XML (xsd) para a lista de tipos de dados internos. Por isso, uma nova versão foi criada: a OWL 2. Essa versão melhora consideravelmente os tipos de dados, além de abordar problemas agudos com expressividade. Um objetivo no desenvolvimento do OWL 2 foi fornecer uma plataforma robusta para o desenvolvimento futuro. Na OWL 2 foram definidos cinco perfis diferentes que podem ser utilizados para fins de representação distintos, são eles:
        
        \begin{itemize}
        	\item Full: o nível mais completo de expressividade de OWL, porém não tem a garantia de decidibilidade computacional;
            \item DL: possui o máximo de expressividade e a garantia computacional de que todas as inferências são computáveis e que têm decidibilidade com as inferências terminando em tempo finito;
            \item EL: adequada para ontologias que definem um grande número de classes e/ou propriedades;
            \item QL: possibilita a inferência em ontologias menos complexas e que possuem dados mais estruturados, como banco de dados relacionais.
            \item RL: é voltado para aplicativos que exigem raciocínio escalável sem sacrificar muita energia com expressividade. Ele é projetado para acomodar aplicativos OWL 2 que podem trocar a expressividade total da linguagem pela eficiência.
        \end{itemize}

\section{Considerações Finais}

Neste capítulo, procurou-se realizar o embasamento necessário para que se possar ter um entendimento claro do tema desta pesquisa e ainda apresentar uma visão geral das técnicas que serão utilizadas para resolução do problema de pesquisa. A maiorias das referências apresentadas têm, como seu foco principal, o cenário brasileiro, que é onde se deseja implementar a proposta.


\chapter{Trabalhos Relacionados}
\label{chapter:trabalhos-relacionados}
Os evidentes problemas do Sistema Único de Saúde trazem impactos negativos para toda a sociedade, individualmente e coletivamente. Na literatura são encontrados diversos trabalhos que identificam lacunas e propõem soluções a fim de melhorar a qualidade da saúde no Brasil. 

Quando se trata de fila de espera, a grande maioria dos trabalhos tem como objetivo analisar o tempo de espera e propor melhorias para diminuição do tempo médio de atendimento dos pacientes. Poucos trabalhos estudam se as filas de espera são realmente justas de acordo com os critérios de priorização de cada caso.

Portanto, foi realizada uma revisão na literatura onde buscou-se identificar os trabalhos que se relacionassem com o objetivo desta pesquisa, estudando as filas de espera do SUS e seus critérios de priorização no atendimento. A pesquisa bibliográfica foi realizada nos repositórios da Scielo, Web of Science, Researchgate, ACM e IEEE Xplore, além da utilização dos buscadores Google Scholar e Portal de Busca Integrada da Universidade de São Paulo. Os termos utilizados para a pesquisa foram:

\begin{itemize}
    	\item Em português: "critérios" AND "priorização" AND ( "filas de espera" OR "lista de espera") AND "saúde".
        \item Em inglês: "criteria" AND "prioritization" AND ("waiting queues" OR "waiting list") AND "health"
\end{itemize}

Após a leitura do titulo de cada trabalho retornado nas buscas, foi realizada a leitura do resumo de cada trabalho que contivesse os termos referentes a esta pesquisa: filas de espera, critérios de priorização e saúde. Em um primeiro momento, para ser considerado como um trabalho relacionado, um trabalho devia discutir os critérios para priorização das filas de espera na área da saúde. Em uma segunda etapa, visto que a primeira etapa de busca não obteve muitos trabalhos, também foram levados em consideração as pequisas que tivessem como abordagem o cenário das filas de espera atualmente no Brasil.

O trabalho de \citeonline{Lippi2018}, faz uma revisão da literatura de itens pertinentes a listas de espera da área da saúde. Ao longo do texto, os autores enfatizam que a literatura brasileira sobre o tema é escassa, embora esse seja discutido na literatura internacional, principalmente na forma de estudos de caso. A título exemplificativo, o trabalho relata que, em novembro de 2017, a busca na base Capes por “espera” (no título e “é exato”) AND “gestão” (qualquer e “é exato”), sem qualquer outro filtro, resulta em 10 resultados, dos quais 3 remetem ao tema e discutem casos de outros países, e 5 não são afetos ao setor da saúde. Nos tópicos seguintes, são apresentados modelos de gestão das filas de espera em outros países e mencionada uma política identificada na Austrália para utilização de critérios de avaliação de prioridade clínica. Com o objetivo de garantir que os pacientes com maior necessidade e potencial de se beneficiar sejam tratados em primeiro lugar. Nessa política, desenvolveu-se ferramentas de pontuação para alguns tipos de cirurgias que se baseiam em fatores clínicos para uma avaliação quantitativa para definir a urgência do procedimento para cada paciente.

No trabalho de \citeonline{FERRER2015},  buscou-se caracterizar um serviço de fisioterapia municipal, avaliar e identificar o perfil dos pacientes em lista de espera e propor estratégias de microrregulação do acesso ao atendimento fisioterapêutico em nível secundário para melhoria da resolutividade do sistema. Identificou-se que 72\% dos pacientes não necessitavam da complexidade de um atendimento fisioterapêutico secundário, ou seja, os encaminhamentos realizados da atenção básica para a especializada em grande parte poderiam ter sidos resolvidos na própria atenção básica. Segundo os autores, isso ocorre devida à baixa  resolutividade na atenção primária, à ausência de coordenação entre as equipes de fisioterapia, à falta de comunicação com os demais profissionais, e aos critérios de triagem e atendimento em nível secundário de atenção. Como não há critérios bem definidos para a realização do encaminhamento aos serviços especializados, cada caso pode ser direcionado ao nível secundário sem ter a real necessidade e de acordo com a prioridade do médico solicitante, não seguindo os critérios de atendimento e priorização do serviço especializado.

Como no último trabalho apresentado, o estudo de \citeonline{Tanabe2018} também analisa os encaminhamentos do atendimento primário ao especializado, porém, na área de odontologia. Após a coleta das fichas de registro dos atendimentos realizados, o CID ou hipótese de diagnóstico de cada paciente foi comparado ao Protocolo de Regulação Ambulatorial em Estomatologia da SES/SC, de forma a identificar, a partir da classificação de risco, a pertinência do referenciamento ao nível secundário de atenção. Como resultado, observou-se que  11,6\% dos encaminhamentos foram classificados como situações de rotina, fato que pode indicar a necessidade de maior capacitação dos profissionais da UBS para o diagnóstico e tratamento desses casos. Ou seja, para efeito de comparação com o último trabalho apresentado, casos que também poderiam ter sidos resolvidos na atenção primária se houvessem critérios de encaminhamento e priorização definidos por especialidades.

No Brasil, o governo federal através do Ministério da Saúde, disponibiliza para as unidades de saúde o \sigla{SISREG}{\textit{Sistema Nacional de Regulação}} \cite{SISREG2017}. Esse é o sistema oficial para regulação dos serviços públicos de saúde no Brasil, com módulos para a regulação ambulatorial e hospitalar. A utilização dessa ferramenta por parte das unidades de saúde não tem caráter compulsório, ou seja, os municípios podem optar pelo uso do SISREG ou outra ferramenta.
No SISREG, há dois perfis principais, que são:

\begin{itemize}
    	\item \textbf{SOLICITANTE} - médico, enfermeiros e dentistas das unidades básicas de saúde que solicitam o atendimento especializado ao paciente.
        \item \textbf{REGULADOR/AUTORIZADOR} - Funcionário da unidade de saúde (médico ou não) a regular todas as solicitações recebidas no sistema e agendar com os serviços especializados de acordo com a agenda divulgada mensalmente.
\end{itemize}

Os procedimentos, contidos no SISREG, são agendados e atendidos sequencialmente, exceto quando o SOLICITANTE relata a necessidade de uma priorização do caso.
Os municípios  determinam como é feita essa regulação, se cada UBS contará com um regulador para análise caso a caso ou se os casos deverão ir para uma central de regulação. 
Diferentemente da proposta deste trabalho, o SISREG não tem como objetivo realizar a regulação computacionalmente, cada caso é analisado manualmente e os critérios utilizados para uma priorização não são transparentes por depender de cada \textbf{REGULADOR/AUTORIZADOR} analisar e inserir no sistema a ordem que achar mais adequada.

Dos trabalhos encontrados, o de \citeonline{BUSS2015} foi o mais próximo ao tema do presente estudo. Ele trata de listas de espera de cirurgias e não de atendimento ambulatorial, mas esses dois tipos de listas têm características similares. No trabalho, é realizado um levantamento na literatura mostrando que os critérios para priorização de listas de espera de cirurgias no SUS são ineficazes e, em alguns casos, inexistentes. Como solução, é realizada uma pesquisa na literatura para se obter um conjunto de características físicas e sociais, de indivíduos que aguardam tratamento cirúrgico, como parâmetros de priorização de casos nas listas de espera.
Após a consolidação dessas características, foi realizada uma pesquisa Delphi aplicada a cirurgiões especialistas atuantes na rede pública de saúde.

Como resultado da pesquisa Delphi, obteve-se a formação de um conjunto de 16 características que, segundo o autor, são aplicáveis como fatores de priorização para qualquer procedimento cirúrgico. As características apresentadas no trabalho, como itens a serem considerados durante uma priorização de algum caso em relação aos demais contidos na lista são:

  \begin{itemize}
    	\item Acesso aos medicamentos
        \item A enfermidade afeta o paciente psicologicamente.
        \item Comorbidades
        \item Gravidade da doença
        \item Idade
        \item Índice de sucesso do procedimento
        \item Ocorrência de dor
        \item Paciente depende de terceiros
        \item Paciente vive sozinho
        \item Privação múltipla 
        \item Responsabilidade sobre o rendimento da família
        \item Sequela aumenta com o tempo
        \item Sequela posterior
        \item Tempo de Espera 
        \item Tempo do início dos sintomas
        \item Velocidade da progressão da doença
    \end{itemize}

% algumas destas características podem ser aplicadas às filas ambulatoriais, mas algumas são específicas para filas cirúrgicas...

O autor relata que esse conjunto de características são globais, para todas as especialidades. Porém é necessário criar uma forma de ordenação entre esses itens, como forma de equacionar o grau de significância de um determinado parâmetro de priorização. É necessário que esse processo leve
em conta para qual procedimento cada característica está sendo utilizada. Ou seja, cada uma deverá ter seu “peso” calculado frente a cada tipo de procedimento.

Foi elaborada uma ontologia de domínio, em formato OWL, como forma de prover um meio de ilustração e de facilitação da difusão das informações acerca do domínio de conhecimento representado.

Como forma de validar o estudo, foi elaborada uma aplicação do modelo posposto para o serviço de Cirurgia Plástica e Queimados do Hospital Universitário Prof. Polydoro Hernane de São Thiago – HU UFSC. Foi utilizado o conjunto de características mencionado e realizada uma pesquisa de método de preferência declarada com os especialistas do hospital para a classificação conforme o grau de importância que cada item do conjunto possui frente ao procedimento cirúrgico.
Foram entregues 6 cartões, cada um simulando características de diferentes pacientes, para que o especialista os ordenasse conforme sua percepção de necessidade do atendimento. A pesquisa conclui com o "peso" de cada característica para priorização e gerenciamento de uma lista de espera para cirurgia plástica e queimados.

Apesar do trabalho não focar diretamente no gerenciamento de listas de espera computacionalmente, mas sim no levantamento do conjunto de características para um melhor gerenciamento (de acordo com critérios previamente definidos por especialistas), ele se assemelha com os objetivos da presente pesquisa. Suas conclusões auxiliam esta pesquisa pois o gerenciamento das listas para atendimento também precisa utilizar parâmetros de acordo com as condições físicas e sociais de cada individuo.



\section{Considerações Finais}

    O capítulo apresenta os trabalhos relacionados, inclusive apresentando o sistema de regulação que é disponibilizado pelo governo para as unidades de saúde no Brasil. É possível analisar o modo da realização da regulação com o sistema e fazer um comparativo com o presente estudo. Posteriormente, é apresentado um trabalho no qual foi identificado um conjunto de características que se deve levar em consideração para o gerenciamento e priorização de casos das listas de espera.
    
    Estes trabalhos foram escolhidos por estarem relacionados com esta monografia, abordando itens da regulação e gerenciamento da fila de espera do sistema único de saúde (SUS). É importante destacar o quão escassos são os estudos sobre a fila de espera no Brasil \cite{URSULA2018}, especialmente estudos que analisem o ordenamento e os critérios usados para manutenção das filas. 
    %Note que o paragrafo seguinte pode até ser verdade, mas não é o objetivo do seu trabalho. Os criterios são da área médica (não são sua contribuição). Você tem que mostrar que pode pegar esses criterios e aplica-los de maneira automática para o gerenciamento da fila.
    Com os trabalhos apresentados, ainda é possível constatar como os critérios de priorização precisam ser estudados e serem propostas novas soluções.

\chapter{Proposta}
\label{chapter:proposta}
Este capítulo apresenta o problema identificado, a proposta de resolução para este problema, o planejamento de ações para alcançar a solução e os resultados que são esperados ao fim da pesquisa.



\section{Metodologia}
\label{sec:methodology}

 Para a resolução do problema apresentado, é proposto a criação de uma ontologia que contemple as propriedades necessárias de pacientes a serem atendidos em uma especialidade médica e um protótipo que utilize esses itens e gerencie uma fila de espera de acordo com critérios preestabelecidos por médicos.
 
 Como forma de obter um primeiro protótipo para resolução de problemas da área da saúde utilizando tecnologias da web semântica, foi desenvolvido uma aplicação com o objetivo de extrair os termos presentes em hemogramas em formato PDF baseados em texto e relacionar cada termo extraído com a terminologia LOINC, permitindo que dados a princípio tabulares, pudessem após essa tratativa, serem compreendidos  pelos computadores.

    Com o intuito de validar a proposta, será desenvolvido um estudo de caso com o Hospital das Clínicas da Faculdade de Medicina de Marília (HCFAMEMA), onde se espera implantar a ontologia e o protótipo. Médicos desse hospital avaliarão o gerenciamento realizado pelo protótipo. Eles informarão qual sua concordância com a fila de atendimento gerada pelo protótipo (de acordo com a quantidade de pacientes que julgarem estar corretamente posicionados) e, para efeito de comparação, também informarão sua concordância com a fila de atendimento atualmente usada pelo hospital (para o mesmo conjunto de pacientes). Nosso objetivo é determinar se o protótipo pode, de fato, gerenciar uma fila de espera de forma tão (ou mais) eficaz que o que é feito atualmente no hospital segundo a visão de especialistas (médicos).
	
	O intuito não é discutir quais devem ser os critérios a serem utilizados pelos médicos para avaliar se um paciente X tem prioridade em relação ao paciente Y, mas sim possibilitar que, a partir de critérios previamente estabelecidos, o computador possa fazer essa triagem de forma tão (ou mais) eficaz que os funcionários que hoje desempenham essa tarefa. O objetivo é permitir um ordenamento mais próximo ao que os médicos aconselham, levando a uma fila de atendimento mais justa e transparente. 
	
	O método para se chegar ao objetivo desta pesquisa é utilizar recursos da web semântica, como ontologias, para que se possa representar características de pacientes que poderão ser utilizados para gerenciar uma fila de espera. Devido as restrições de tempo de um trabalho de mestrado, o protótipo desenvolvido vai se restringir a uma especialidade médica específica. Isso visa reduzir o escopo da ontologia a ser desenvolvida e o tempo de desenvolvimento do protótipo.

    \section{Cronograma}

	Considerando um tempo de 25 meses para o desenvolvimento da pesquisa proposta e a numeração das tarefas apresentadas na \autoref{sec:methodology}, o cronograma das atividades se apresenta na Tabela 1 %\autoref{tab:cronogram}.
	
	  \begin{enumerate}
    	\item Conclusão da validação do exame de proficiência em língua estrangeira;
        \item Integralização dos créditos obrigatórios;
        \item Mapeamento da literatura;
        \item Elaboração e apresentação dos artefatos relacionados ao exame de qualificação;
         \item Desenvolvimento da ontologia OWL que represente indicadores para classificação de prioridades para o atendimento de uma especialidade médica;
        \item Desenvolvimento e testes de um protótipo que utilize a ontologia desenvolvida e gerencie uma fila de espera utilizando critérios definidos;
        \item Elaboração e submissão de artigos;
        \item Elaboração e apresentação dos artefatos relacionados a defesa;
    \end{enumerate}
    
    \begin{table}[htbp]
    	\centering
        \caption{Cronograma das atividades da pesquisa.}
        \label{tab:cronogram}
        \includegraphics[width=1\linewidth]{images/cronogram}
        \fautor
    \end{table}
    
    \pagebreak
    
   \section{ Resultados Esperados}

	Ao final desta pesquisa espera-se as seguintes contribuições:
	
	\begin{itemize}
		\item Uma ontologia OWL que represente indicadores para classificação de prioridades para o atendimento de uma especialidade médica;
        \item  Um protótipo que gerencie filas de espera computacionalmente e se mostre  mais eficaz que o gerenciamento atual do HC FAMEMA;
        \item Estudo de caso que mostre que o modelo proposto pode auxiliar na gestão de filas do Sistema Único de Saúde (no HC FAMEMA) e ter benefícios se comparado a outras formas de gerenciamento;
	\end{itemize}



% ---
% Finaliza a parte no bookmark do PDF, para que se inicie o bookmark na raiz
% ---
\bookmarksetup{startatroot}% 
% ---

% ----------------------------------------------------------
% ELEMENTOS PÓS-TEXTUAIS
% ----------------------------------------------------------
\postextual

% ----------------------------------------------------------
% Referências bibliográficas
% ----------------------------------------------------------
\bibliography{references}

\end{document}