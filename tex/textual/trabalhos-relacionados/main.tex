Os evidentes problemas do Sistema Único de Saúde trazem impactos negativos para toda a sociedade, individualmente e coletivamente. Na literatura são encontrados diversos trabalhos que identificam lacunas e propõem soluções a fim de melhorar a qualidade da saúde no Brasil. 

Quando se trata de fila de espera, a grande maioria dos trabalhos tem como objetivo analisar o tempo de espera e propor melhorias para diminuição do tempo médio de atendimento dos pacientes. Poucos trabalhos estudam se as filas de espera são realmente justas de acordo com os critérios de priorização de cada caso.

Portanto, foi realizada uma revisão na literatura onde buscou-se identificar os trabalhos que se relacionassem com o objetivo desta pesquisa, estudando as filas de espera do SUS e seus critérios de priorização no atendimento. A pesquisa bibliográfica foi realizada nos repositórios da Scielo, Web of Science, Researchgate, ACM e IEEE Xplore, além da utilização dos buscadores Google Scholar e Portal de Busca Integrada da Universidade de São Paulo. Os termos utilizados para a pesquisa foram:

\begin{itemize}
    	\item Em português: "critérios" AND "priorização" AND ( "filas de espera" OR "lista de espera") AND "saúde".
        \item Em inglês: "criteria" AND "prioritization" AND ("waiting queues" OR "waiting list") AND "health"
\end{itemize}

Após a leitura do titulo de cada trabalho retornado nas buscas, foi realizada a leitura do resumo de cada trabalho que contivesse os termos referentes a esta pesquisa: filas de espera, critérios de priorização e saúde. Em um primeiro momento, para ser considerado como um trabalho relacionado, um trabalho devia discutir os critérios para priorização das filas de espera na área da saúde. Em uma segunda etapa, visto que a primeira etapa de busca não obteve muitos trabalhos, também foram levados em consideração as pequisas que tivessem como abordagem o cenário das filas de espera atualmente no Brasil.

O trabalho de \citeonline{Lippi2018}, faz uma revisão da literatura de itens pertinentes a listas de espera da área da saúde. Ao longo do texto, os autores enfatizam que a literatura brasileira sobre o tema é escassa, embora esse seja discutido na literatura internacional, principalmente na forma de estudos de caso. A título exemplificativo, o trabalho relata que, em novembro de 2017, a busca na base Capes por “espera” (no título e “é exato”) AND “gestão” (qualquer e “é exato”), sem qualquer outro filtro, resulta em 10 resultados, dos quais 3 remetem ao tema e discutem casos de outros países, e 5 não são afetos ao setor da saúde. Nos tópicos seguintes, são apresentados modelos de gestão das filas de espera em outros países e mencionada uma política identificada na Austrália para utilização de critérios de avaliação de prioridade clínica. Com o objetivo de garantir que os pacientes com maior necessidade e potencial de se beneficiar sejam tratados em primeiro lugar. Nessa política, desenvolveu-se ferramentas de pontuação para alguns tipos de cirurgias que se baseiam em fatores clínicos para uma avaliação quantitativa para definir a urgência do procedimento para cada paciente.

No trabalho de \citeonline{FERRER2015},  buscou-se caracterizar um serviço de fisioterapia municipal, avaliar e identificar o perfil dos pacientes em lista de espera e propor estratégias de microrregulação do acesso ao atendimento fisioterapêutico em nível secundário para melhoria da resolutividade do sistema. Identificou-se que 72\% dos pacientes não necessitavam da complexidade de um atendimento fisioterapêutico secundário, ou seja, os encaminhamentos realizados da atenção básica para a especializada em grande parte poderiam ter sidos resolvidos na própria atenção básica. Segundo os autores, isso ocorre devida à baixa  resolutividade na atenção primária, à ausência de coordenação entre as equipes de fisioterapia, à falta de comunicação com os demais profissionais, e aos critérios de triagem e atendimento em nível secundário de atenção. Como não há critérios bem definidos para a realização do encaminhamento aos serviços especializados, cada caso pode ser direcionado ao nível secundário sem ter a real necessidade e de acordo com a prioridade do médico solicitante, não seguindo os critérios de atendimento e priorização do serviço especializado.

Como no último trabalho apresentado, o estudo de \citeonline{Tanabe2018} também analisa os encaminhamentos do atendimento primário ao especializado, porém, na área de odontologia. Após a coleta das fichas de registro dos atendimentos realizados, o CID ou hipótese de diagnóstico de cada paciente foi comparado ao Protocolo de Regulação Ambulatorial em Estomatologia da SES/SC, de forma a identificar, a partir da classificação de risco, a pertinência do referenciamento ao nível secundário de atenção. Como resultado, observou-se que  11,6\% dos encaminhamentos foram classificados como situações de rotina, fato que pode indicar a necessidade de maior capacitação dos profissionais da UBS para o diagnóstico e tratamento desses casos. Ou seja, para efeito de comparação com o último trabalho apresentado, casos que também poderiam ter sidos resolvidos na atenção primária se houvessem critérios de encaminhamento e priorização definidos por especialidades.

No Brasil, o governo federal através do Ministério da Saúde, disponibiliza para as unidades de saúde o \sigla{SISREG}{\textit{Sistema Nacional de Regulação}} \cite{SISREG2017}. Esse é o sistema oficial para regulação dos serviços públicos de saúde no Brasil, com módulos para a regulação ambulatorial e hospitalar. A utilização dessa ferramenta por parte das unidades de saúde não tem caráter compulsório, ou seja, os municípios podem optar pelo uso do SISREG ou outra ferramenta.
No SISREG, há dois perfis principais, que são:

\begin{itemize}
    	\item \textbf{SOLICITANTE} - médico, enfermeiros e dentistas das unidades básicas de saúde que solicitam o atendimento especializado ao paciente.
        \item \textbf{REGULADOR/AUTORIZADOR} - Funcionário da unidade de saúde (médico ou não) a regular todas as solicitações recebidas no sistema e agendar com os serviços especializados de acordo com a agenda divulgada mensalmente.
\end{itemize}

Os procedimentos, contidos no SISREG, são agendados e atendidos sequencialmente, exceto quando o SOLICITANTE relata a necessidade de uma priorização do caso.
Os municípios  determinam como é feita essa regulação, se cada UBS contará com um regulador para análise caso a caso ou se os casos deverão ir para uma central de regulação. 
Diferentemente da proposta deste trabalho, o SISREG não tem como objetivo realizar a regulação computacionalmente, cada caso é analisado manualmente e os critérios utilizados para uma priorização não são transparentes por depender de cada \textbf{REGULADOR/AUTORIZADOR} analisar e inserir no sistema a ordem que achar mais adequada.

Dos trabalhos encontrados, o de \citeonline{BUSS2015} foi o mais próximo ao tema do presente estudo. Ele trata de listas de espera de cirurgias e não de atendimento ambulatorial, mas esses dois tipos de listas têm características similares. No trabalho, é realizado um levantamento na literatura mostrando que os critérios para priorização de listas de espera de cirurgias no SUS são ineficazes e, em alguns casos, inexistentes. Como solução, é realizada uma pesquisa na literatura para se obter um conjunto de características físicas e sociais, de indivíduos que aguardam tratamento cirúrgico, como parâmetros de priorização de casos nas listas de espera.
Após a consolidação dessas características, foi realizada uma pesquisa Delphi aplicada a cirurgiões especialistas atuantes na rede pública de saúde.

Como resultado da pesquisa Delphi, obteve-se a formação de um conjunto de 16 características que, segundo o autor, são aplicáveis como fatores de priorização para qualquer procedimento cirúrgico. As características apresentadas no trabalho, como itens a serem considerados durante uma priorização de algum caso em relação aos demais contidos na lista são:

  \begin{itemize}
    	\item Acesso aos medicamentos
        \item A enfermidade afeta o paciente psicologicamente.
        \item Comorbidades
        \item Gravidade da doença
        \item Idade
        \item Índice de sucesso do procedimento
        \item Ocorrência de dor
        \item Paciente depende de terceiros
        \item Paciente vive sozinho
        \item Privação múltipla 
        \item Responsabilidade sobre o rendimento da família
        \item Sequela aumenta com o tempo
        \item Sequela posterior
        \item Tempo de Espera 
        \item Tempo do início dos sintomas
        \item Velocidade da progressão da doença
    \end{itemize}

% algumas destas características podem ser aplicadas às filas ambulatoriais, mas algumas são específicas para filas cirúrgicas...

O autor relata que esse conjunto de características são globais, para todas as especialidades. Porém é necessário criar uma forma de ordenação entre esses itens, como forma de equacionar o grau de significância de um determinado parâmetro de priorização. É necessário que esse processo leve
em conta para qual procedimento cada característica está sendo utilizada. Ou seja, cada uma deverá ter seu “peso” calculado frente a cada tipo de procedimento.

Foi elaborada uma ontologia de domínio, em formato OWL, como forma de prover um meio de ilustração e de facilitação da difusão das informações acerca do domínio de conhecimento representado.

Como forma de validar o estudo, foi elaborada uma aplicação do modelo posposto para o serviço de Cirurgia Plástica e Queimados do Hospital Universitário Prof. Polydoro Hernane de São Thiago – HU UFSC. Foi utilizado o conjunto de características mencionado e realizada uma pesquisa de método de preferência declarada com os especialistas do hospital para a classificação conforme o grau de importância que cada item do conjunto possui frente ao procedimento cirúrgico.
Foram entregues 6 cartões, cada um simulando características de diferentes pacientes, para que o especialista os ordenasse conforme sua percepção de necessidade do atendimento. A pesquisa conclui com o "peso" de cada característica para priorização e gerenciamento de uma lista de espera para cirurgia plástica e queimados.

Apesar do trabalho não focar diretamente no gerenciamento de listas de espera computacionalmente, mas sim no levantamento do conjunto de características para um melhor gerenciamento (de acordo com critérios previamente definidos por especialistas), ele se assemelha com os objetivos da presente pesquisa. Suas conclusões auxiliam esta pesquisa pois o gerenciamento das listas para atendimento também precisa utilizar parâmetros de acordo com as condições físicas e sociais de cada individuo.



\section{Considerações Finais}

    O capítulo apresenta os trabalhos relacionados, inclusive apresentando o sistema de regulação que é disponibilizado pelo governo para as unidades de saúde no Brasil. É possível analisar o modo da realização da regulação com o sistema e fazer um comparativo com o presente estudo. Posteriormente, é apresentado um trabalho no qual foi identificado um conjunto de características que se deve levar em consideração para o gerenciamento e priorização de casos das listas de espera.
    
    Estes trabalhos foram escolhidos por estarem relacionados com esta monografia, abordando itens da regulação e gerenciamento da fila de espera do sistema único de saúde (SUS). É importante destacar o quão escassos são os estudos sobre a fila de espera no Brasil \cite{URSULA2018}, especialmente estudos que analisem o ordenamento e os critérios usados para manutenção das filas. 
    %Note que o paragrafo seguinte pode até ser verdade, mas não é o objetivo do seu trabalho. Os criterios são da área médica (não são sua contribuição). Você tem que mostrar que pode pegar esses criterios e aplica-los de maneira automática para o gerenciamento da fila.
    Com os trabalhos apresentados, ainda é possível constatar como os critérios de priorização precisam ser estudados e serem propostas novas soluções.