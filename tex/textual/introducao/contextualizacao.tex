A fila de espera faz parte do nosso cotidiano, enfrentamos filas para diversos objetivos, seja no supermercado, em Bancos e até mesmo para um atendimento médico. No caso dos serviços de saúde, a longa espera pode propiciar o sofrimento do paciente, reduzir as possibilidades de cura, permitir o agravamento das enfermidades ou a extensão das sequelas e até determinar risco de morte \cite{Gazzinelli2014}. 

No Sistema Único de Saúde(SUS), marcar consulta com especialista é o maior problema para os brasileiros \cite{CFM2018}. Não há um modelo claro de regulação do SUS, ficando a cargo de cada município gerir as filas de espera para os atendimentos especializados. A gestão destas filas ocorre de maneira manual, em muitos casos por funcionários das unidades de saúde que não obtém de um conhecimento  amplo para identificar e comparar casos de cada enfermidade. Neste cenário, os funcionários adotam um agendamento das consultas, predominantemente, pela ordem de chegada das fichas de encaminhamento na recepção, exceto quando o médico sinaliza a urgência no atendimento do caso, como relatado no trabalho de \cite{SOUZA2014}.

Apesar de fazer parte do dia-a-dia dos médicos que trabalham em serviço público, o problema da fila de espera é muito pouco abordado pela comunidade médica e científica, talvez por parecer tratar-se de uma discussão que não pertence aos meios acadêmicos e sim às instâncias governamentais. Todavia é preciso salientar que o acesso equitativo, justo e universal aos serviços de saúde deve ser uma preocupação constante não só do governo como de todos os profissionais envolvidos no atendimento à rede pública. Muito pode ser feito em nível local a respeito das filas de espera. \cite{KRISHNAMURTI2005}

O tempo de espera é um indicador da qualidade dos serviços, por estar relacionado à capacidade de resposta do sistema às necessidades de atenção à saúde da população \cite{Gazzinelli2014}. 

A tecnologia é muito importante para a gestão de saúde,
ultrapassando o processamento-padrão de dados para funções administrativas comuns em todas as organizações, tais como recursos humanos, folhas de pagamento, sistemas de contabilidade, entre outros, e agora desempenha um papel fundamental tanto no cuidado ao paciente, na interpretação do eletrocardiograma, como em escalas de trabalho, prescrição, relatório de resultados e sistemas de prevenção \cite{PINOCHET2014}. O computador pode auxiliar no processamento de informações de saúde, para \cite{MATTOS1978}, as tarefas realizadas pelo computador, desde que corretamente programado, são praticamente isentas de erro, o que não ocorre no caso do processamento manual (realizado por pessoas). O trabalho ainda relata que o principal motivo dessa perfeição é que a máquina não é sensível aos fatores que reduzem a qualidade do trabalho humano, quais sejam:

  \begin{enumerate}
    	\item a fadiga - que faz com que o número de erros cometidos involuntariamente aumente ao fim do turno de trabalho;
        \item Dos problemas pessoais - uma preocupação com a família, por exemplo, perturba bastante o bom desempenho do funcionário;
        \item o desajuste com a empresa - um empregado mal pago, ou ameaçado de ser despedido, "vinga-se" da empresa sabotando o sistema de informações do qual participa;
        \item interesses pessoais - o funcionário procura adulterar o sistema em beneficio próprio. Tal é o caso, por exemplo, da emissão de uma ordem de serviço para a instalação de um telefone em uma loja comercial, passando na frente de centenas de outros clientes, por um empregado" que recebe uma propina do proprietário dessa loja;
        \item desorganização e falta de planejamento - a incapacidade de um chefe pode desorientar seus subordinados, reduzindo a qualidade de seu serviço.
    \end{enumerate}
    
O trabalho apresenta os itens com uma visão mais empresarial, porém, é possível ter um entendimento com um olhar sobre os funcionários gestores das filas de espera para o encaminhamento especializado. O item 4 por exemplo, é perfeitamente comparável a de funcionários que priorizam o atendimento de amigos e parentes na gestão destas filas, conforme citado no trabalho de \cite{KRISHNAMURTI2005}

A Web Semântica é uma destas tecnologias que podem auxiliar processos na área da saúde. Atrvés da OWL - Ontology Web Language, criada pela World Wide Web Consortium (W3C), é possível tornar as informações, antes apenas compreensível por humanos, em informações passíveis de interpretação por computadores. O gerenciamento das filas de espera por parte dos computadores, pode contribuir para se alcançar filas isentas de fatores humanos (descritos na lista anterior), ainda possibilitar a redução do tempo de espera para os pacientes que precisam de um atendimento imediato de acordo com suas características físicas e sociais. 

Como uma instância destas lacunas é possível citar o caso do Hospital das Clínicas da Faculdade de Medicina de Marília(FAMEMA), onde o problema da fila de espera ocorre em casos de retorno de um paciente a uma especialidade, cuja fila é denominada como "Demanda". Na Demanda, é utilizado o método tradicional de fila(First In, First Out - FIFO), gerando "filas sem fim", nas quais os pacientes chegam a esperar mais de oito anos por um atendimento. No gerenciamento da Demanda, ainda não é levado em consideração o quadro clínico do paciente, tais como o diagnóstico, gravidade, idade e a capacidade de um paciente se recuperar de uma enfermidade no caso da urgência no seu retorno. A demanda é distribuída por profissional e não por especialidade, necessitando que em casos de desligamento de algum profissional, a distribuição manual da fila para outros médicos.

Um efeito colateral do gap entre a entrada do paciente na Demanda e o seu efetivo atendimento é o absenteísmo. Por vários motivos: o problema inicial para o qual o paciente procurou atendimento não existe mais, foi atendido por outro estabelecimento/profissional de saúde, o paciente não foi localizado, o paciente veio a óbito, entre outros. 

