A fila de espera faz parte do nosso cotidiano, enfrentamos filas para diversos objetivos, seja no supermercado, em bancos e até mesmo para um atendimento médico. No caso dos serviços de saúde, a longa espera pode ocasionar o sofrimento do paciente, reduzir as possibilidades de cura, permitir o agravamento das enfermidades ou a extensão das sequelas e até determinar risco de morte \cite{Gazzinelli2014}. 

No \sigla{SUS}{\textit{Sistema Único de Saúde}}, a maior dificuldade de acesso para os brasileiros é marcar consulta com especialista \cite{CFM2018}. Não há um modelo claro de regulação do SUS, ficando a cargo de cada município gerir as filas de espera para os atendimentos especializados. A gestão dessas filas ocorre de maneira manual, em muitos casos por funcionários das unidades de saúde que não obtém de um conhecimento amplo para identificar e comparar os casos de cada enfermidade. Nesse cenário, os funcionários realizam o agendamento das consultas, predominantemente, pela ordem de chegada das fichas de encaminhamento na recepção, exceto quando o médico sinaliza a urgência no atendimento do caso \cite{SOUZA2014}.

Como não existe uma politica federal e obrigatória, estados e municípios têm a liberdade de criar suas centrais de regulação. Um exemplo disso, no segundo semestre de 2010 foi criada a \sigla{CROSS}{\textit{Central de Regulação de Ofertas de Serviços de Saúde }} no estado de São Paulo. Central que dispõe de um sistema informatizado, onde é possível que os serviços secundários e terciários possam lançar as vagas para atendimento, e os serviços primários (unidades básicas de saúde) possam agendar os pacientes.

Apesar de fazer parte do dia-a-dia dos médicos que trabalham no serviço público, o problema da fila de espera é muito pouco abordado pela comunidade médica e científica, talvez por parecer tratar-se de uma discussão que não pertence aos meios acadêmicos e sim às instâncias governamentais. Todavia é preciso salientar que o acesso equitativo, justo e universal aos serviços de saúde deve ser uma preocupação constante não só do governo mas de todos os profissionais envolvidos no atendimento da rede pública. Muito pode ser feito em nível local a respeito das filas de espera \cite{KRISHNAMURTI2005}.

O tempo de espera é um indicador da qualidade dos serviços, por estar relacionado à capacidade de resposta do sistema às necessidades de atenção à saúde da população \cite{Gazzinelli2014}. 
    
Não há controle das listas de espera ou uma estratégia padrão para organizá-las. Ficando a critério das instituições a escolha da forma de sistematização, se por especialidade, prestador ou por data, gerando então diversas filas e informações que dificultam inclusive o monitoramento e gerenciamento das mesmas \cite{URSULA2018}.
    
Como cada município pode fazer a gestão da fila, nem sempre é um especialista que elabora a lista de espera, tendo como efeito a possibilidade de não utilizar critérios corretos para a ordem de atendimento. Algumas possíveis soluções para o problema da fila de espera merecem discussão. Medidas que aumentem a oferta de serviços especializados estão, frequentemente, entre as principais estratégias para a redução do tempo de espera. Mas entende-se que simplesmente aumentar a oferta de consultas, visando reduzir a lista de espera, apenas encorajaria o número de encaminhamentos \cite{Gazzinelli2014}.

A tecnologia é muito importante para a gestão de saúde, ultrapassando o processamento-padrão de dados para funções administrativas comuns em todas as organizações, tais como recursos humanos ou folhas de pagamento. Ela desempenha um papel fundamental tanto no cuidado ao paciente (como na interpretação de um eletrocardiograma) como em tarefas auxiliares como escalas de trabalho, prescrição, relatórios de resultados e sistemas de prevenção \cite{Pinochet2014}. O computador pode auxiliar no processamento de informações de saúde, para \citeonline{MATTOS1978}, as tarefas realizadas pelo computador, desde que corretamente programadas, são praticamente isentas de erro, o que não ocorre no caso do processamento manual (realizado por pessoas). O trabalho ainda relata que o principal motivo dessa perfeição é que a máquina não é sensível aos fatores que reduzem a qualidade do trabalho humano, quais sejam:

  \begin{enumerate}
    	\item a fadiga - que faz com que o número de erros cometidos involuntariamente aumente ao fim do turno de trabalho;
        \item Dos problemas pessoais - uma preocupação com a família, por exemplo, perturba bastante o bom desempenho do funcionário;
        \item o desajuste com a empresa - um empregado mal pago, ou ameaçado de ser despedido, "vinga-se" da empresa sabotando o sistema de informações do qual participa;
        \item interesses pessoais - o funcionário procura adulterar o sistema em beneficio próprio. Tal é o caso, por exemplo, da emissão de uma ordem de serviço para a instalação de um telefone em uma loja comercial, passando na frente de centenas de outros clientes, por um empregado que recebe uma propina do proprietário dessa loja;
        \item desorganização e falta de planejamento - a incapacidade de um chefe pode desorientar seus subordinados, reduzindo a qualidade de seu serviço.
    \end{enumerate}
    
Esse trabalho apresenta os itens com uma visão mais empresarial, porém, é possível ter um entendimento com um olhar sobre os funcionários gestores das filas de espera para o encaminhamento especializado. O item 4, por exemplo, é perfeitamente comparável a de funcionários que priorizam o atendimento de amigos e parentes na gestão dessas filas, conforme citado no trabalho de \citeonline{KRISHNAMURTI2005}.

A Web Semântica é uma das tecnologias que podem auxiliar processos na área da saúde. Através da \sigla{OWL}{\textit{Ontology Web Language}}, criada pelo \sigla{W3C}{\textit{World Wide Web Consortium}}, é possível tornar as informações, antes apenas compreensíveis por humanos, em informações passíveis de interpretação por computadores. O gerenciamento das filas de espera por parte dos computadores, pode contribuir para se alcançar filas isentas de fatores humanos (descritos na lista anterior) e ainda possibilitar a redução do tempo de espera para os pacientes que precisam de um atendimento imediato de acordo com suas características físicas e sociais. 

No Hospital das Clínicas da \sigla{FAMEMA}{\textit{Faculdade de Medicina de Marília}} existe o problema da fila de espera em casos de retorno de um paciente a uma especialidade, cuja fila é denominada como "Demanda". Na Demanda, é utilizado o método tradicional de fila (First In, First Out - FIFO), gerando "filas sem fim", nas quais os pacientes chegam a esperar por anos por um atendimento. No gerenciamento da Demanda, não é levado em consideração o quadro clínico do paciente, tais como o diagnóstico, gravidade, idade e a capacidade de um paciente se recuperar de uma enfermidade no caso da urgência no seu retorno. A Demanda é distribuída por profissional de saúde e não por especialidade médica, como cada caso da fila está atrelado a um médico, em caso de desligamento de algum profissional ou alguma ausência temporária, é necessária uma redistribuição manual da fila para outros profissionais.

Um efeito colateral do \textit{gap} entre a entrada do paciente na Demanda e o seu efetivo atendimento é o absenteísmo. Ele ocorre por vários motivos: o problema inicial para o qual o paciente procurou atendimento não existe mais, ele foi atendido por outro estabelecimento ou profissional de saúde, o paciente não foi localizado, o paciente veio a óbito, entre outros. 

