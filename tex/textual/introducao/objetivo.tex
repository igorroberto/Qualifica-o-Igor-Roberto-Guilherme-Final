\section{Objetivos}
    O objetivo do trabalho é analisar as filas de espera do Sistema Único de Saúde(SUS) mais precisamente no encaminhamento do atendimento básico para o atendimento especializado e propor um modelo de gerenciamento destas filas computacionalmente com o intuito de tornar-las mais coerentes com a prioridade de atendimento definida por especialistas(Médicos) sem a necessidade da alocação de um profissional dedicada a esta função. 
    
    Ainda pretende-se obter filas mais justas,  minimizando a possibilidade de indivíduos serem atendidos prioritariamente por terem acesso a gestores destas filas, como citado no trabalho de \cite{KRISHNAMURTI2005}. Objetiva-se que os critérios de ordenamento da fila sejam mais transparentes, e que o posicionamento de cada paciente seja feito com isenção, levando em consideração apenas a situação de cada um.

\subsection{Objetivos específicos}

	Para alcançar os objetivos apresentados na seção anterior, foram definidos os objetivos específicos:
    
    \begin{itemize}
    	\item Criação de uma ontologia de domínio no formato OWL que represente indicadores para classificação de prioridades para o atendimento de uma especialidade médica.
        \item Desenvolvimento de um protótipo que utilize a ontologia e de acordo com critérios médicos gerencie uma fila de espera.
        \item Acompanhamento da avaliação dos médicos do Hospital das Clínicas da Faculdade de Medicina de Marília do protótipo com o intuito de verificar se o gerenciamento computacional se sobressai se comparado com o gerenciamento atual do HC.
    \end{itemize}

