\section{Objetivos}
    O objetivo deste trabalho é analisar as filas de espera do Sistema Único de Saúde (SUS), mais precisamente no encaminhamento do atendimento básico para o atendimento especializado, e propor um modelo computacional para o gerenciamento dessas filas, com o intuito de torná-las mais coerentes com a prioridade de atendimento definida por especialistas (médicos), sem a necessidade da alocação de um profissional dedicado à essa função. 
    
    Ainda pretende-se obter filas mais justas,  minimizando a possibilidade de indivíduos serem atendidos prioritariamente por terem acesso aos gestores dessas filas, como citado no trabalho de \citeonline{KRISHNAMURTI2005}. Objetiva-se que os critérios de ordenamento da fila sejam mais transparentes, e que o posicionamento de cada paciente seja feito com isenção, levando em consideração apenas a situação física e social de cada um.

\subsection{Objetivos específicos}

	Para alcançar os objetivos apresentados na seção anterior, foram definidos os objetivos específicos:
    
    \begin{itemize}
    	\item Criação de uma ontologia de domínio no formato OWL que represente indicadores para classificação de prioridades para o atendimento de uma especialidade médica.
        \item Desenvolvimento de um protótipo que utilize a ontologia e, de acordo com critérios médicos, gerencie uma fila de espera.
        \item Acompanhamento da avaliação dos médicos do Hospital das Clínicas da Faculdade de Medicina de Marília do protótipo com o intuito de verificar como o gerenciamento computacional se  sobressai se comparado ao gerenciamento atual do HC.
    \end{itemize}
